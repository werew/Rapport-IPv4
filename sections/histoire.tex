\section{Histoire d'IPv4}
\label{sec:hist}
% Daniel intro

Une forte demande de la part des universités et centres de recherche aux
Etats-Unis a donne lieu a la creation et la mise en ouvre d'un nouveau concept
de reseau permettant d'interconnecter les différentes structures de facon
efficace afin de partager les informations, et d'autre part, de faire des
expérimentations sur les réseaux.
\\
En effet jusqu'alors les reseau informatiques utilisent les memes principes des
reseau telephoniques: la commutation de circuits, ce qui n'etait pas tres 
efficace en terme de resources et materiel\footnote {
Dans une reseau a commutation de circuits chaque peripherique pouvent utiliser
un seul lien a la fois (ciruit): lors d'une trasmission  un lien etait reserve
pour tout la dure de la communication et rendait donc la peripherique
inutiilisable pour des trasmission vers un different destinataire (c'est le meme
principe des reseau telephoniques)
http://www.tcpipguide.com/free/t\_CircuitSwitchingandPacketSwitchingNetworks.htm .}.
\\
Pendant les années 1960 le concept de reseau a paquets commute a ete invente et mise en 
pratique d'abord dans la reseau du NPL (UK National Physical Laboratory) et
puis dans l'agence americaine ARPA ({\it Advanced Research Project Agency})
\footnote {http://www.livinginternet.com/i/iw\_packet\_inv.htm}.
\\
Dans une reseau a commutation de paquets ({\it Packet Switching}) la connection
entre deux machines n'est pas continue mais elle est coupe en plusieurs
paquets.  L'abandon d'une connexion continue a permit de se passer de la
reservation d'un lien (circuit) dediee ce qui comporte la possibilite de
envoier et recevoir en meme temps paquet vers different destinataires (un peu
comme une boite a lettre). 
\\
La reseau créé au sein de l'agence gouvernative ARPA et baptisee ARPANET, est une
des premieres reseaux a fonctionner sur la base de paquets. Le principe de
communication par paquet est de découper l'information à transmettre en de plus
petits paquets qui peuvent chacun prendre un chemin différent pour arriver à
destination.  Avant ARPANET, la communication réseau était basée sur la
communication par circuit électrique dont les informations étaient envoyées en
continue dans un seul morceau. Dans ce sens ARPANET a posee la base a partir de 
laquelle l'internet a ete cree. 
\\
L'ARPA (aujourd'hui DARPA: {\it Defence Advanced Research Project Agency}) est
une agence de recherche créée par le département américain de la défense en
1957 afin de développer de nouvelles technologies à usage militaire.  ARPANET,
la reseau mis en place par ARPA, a été constitué comme une toile reliants
plusieurs serveurs. Chaque serveur est un noeud et peut stocker, traiter ou
servir de relais. Ainsi, il existe plusieurs chemins pour accéder à un noeud et
lorsqu'un noeud est hors service, il est toujours possible de rejoindre le
noeud destinataire en passant par un autre chemin: une des caracteristiques
plus interessant de ARPANET a été une certaine robustesse, ARPANET ne dépendait
pas d'un centre névralgique qui aurait pu être détruit en cas
d'attaque\footnote { Il faut par contre noter que l'hypothese qui affirme que
ARPANET ait ete construit dans le but de creer une reseau resistante aux
attaques nucleaires a ete demystifie par le {\it Internet Society}: 
http://www.internetsociety.org/internet/what-internet/history-internet/brief-history-internet
.}
\\
Ce réseau se développa est il compta 23 noeuds en 1971 et en 1977
il en compta 111. Afin d'uniformiser ce réseau, Vint Cerf et Bob Kahn on
introduit la première version du protocol TCP.  Historiquement, les protocoles
IP constituaient la partie du protocole TCP qui s'occupe de la transmission en
mode sans connexion. La transmission en mode sans connexion est une
transmission de donnée dans laquelle chaque paquet contient l'adresse de
destination. Ceci permet une transmission  du paquet sans que les deux hôtes
soient obligés d'établir une connexion auparavant. Cette version est ce qu'on
aurait pu nommer l'IPv1 et elle est documentée dans la RFC 675. Cette version
fut modifiée et publiée en 1977. Elle correspond à la deuxième version de TCP
(IPv2). 
\\
Initialement, le protocole TCP avait deux fonctions: premièrement, il devait
permettre une transmission fiable d'informations entre deux hôtes en plus
il devait
également servir en tant que protocole de routage et de packaging. 
Cependant,
pour être cohérant avec le modèle en couche, qui différencie la fiabilité
(couche transport) et le routage (couche réseau), il fut décidé en 1978 de
diviser le protocole TCP\footnote {
{\it "We are screwing up in our design of internet protocols by violating the
principle of layering. Specifically we are trying to use TCP to do two
things: serve as a host level end to end protocol, and to serve as an
internet packaging and routing protocol. These two things should be
provided in a layered and modular way. I suggest that a new distinct
internetwork protocol is needed, and that TCP be used strictly as a host
level end to end protocol." } - IEN 2 (Comments on Internet Protocol and TCP)
}.
Le protocole TCP ne s'occupe maintenant plus que de la partie transport. La
partie réseau a été prise en charge par les protocoles IP.  C'est finalement le
1er janvier 1983 que l'ARPANET adopte les protocoles TCP/IP et donc l'IPv4. 
