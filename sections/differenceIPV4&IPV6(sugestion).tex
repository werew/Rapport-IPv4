\section{Différence IPV4 & IPV6}

% MOSSI

Il est important de comprendre que l'IPv6 est beaucoup plus qu'une extension de l'adressage IPv4. IPv6, d'abord défini dans la RFC 2460, est une mise en oeuvre complète de la couche réseau de la pile TCP/IP et il couvre beaucoup plus que l'extension de l'espace d'adressage simple à partir de 32 à 128 bits (le mécanisme qui augmente la capacité d'IPv6 à allouer presque un nombre illimitée d'adresses à tous les appareils dans le monde pour les années à venir).IPv6 offre de nombreuses améliorations par rapport à IPv4, et le tableau ci-dessous compare le fonctionnement de IPv4 et de IPv6.


	Systéme de routage beaucoup plus efficace. Les paquets IPv6 ne sont plus fragmenté par les routeurs.
	
	La qualité de service(QoS) intégrée. Alors que IPv4 n'a aucun moyen de distinguer les paquets sensibles au retard de transferts de données en vrac, ce qui nécessite de nombreuses solutions de contournement, mais IPv6 le fait.
	
	L'élimination du NAT pour élargir les espaces d'adressage. IPv6 augmente la taille de l'adresse IPv4 de 32 bits (environ 4 milliards) à 128 bits (suffisamment pour chaque molécule dans le système solaire).
	
	La sécurité de la couche réseau intégré (IPsec). La sécurité a toujours été un défi en IPv4, mais elle est une partie intégrante de l'IPv6.
	
	L'autoconfiguration d'adresse pour l'administration réseau plus facile. De nombreux installations IPv4 ont été compliquées par le routeur par défaut manuelle et l'attribution d'adresse. IPv6 gère cela de manière automatisée.
	
	L'amélioration de la structure d'en-tête permet d'alléger le traitement. La plupart des champs dans l'en-tête IPv4 étaient facultatifs et utilisés fréquemment. IPv6 élimine ces champs (les options sont traitées différemment).


