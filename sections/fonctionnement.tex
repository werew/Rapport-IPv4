
% Mossi
\subsection{Fonctionnement et Adressage de IPV4}
Une adresse ip est un nombre de 32 bits codé sur 4 octets dont 1 octet égal à 8
bits séparés par un point. Souvent cette adresse est écrite avec des valeurs
décimales.  Il est aussi possible de la saisir sous forme binaire quand c’est
indispensable. Chaque nombre étant compris entre 0 et 255. ou en binaire entre
00000000 et 11111111.  Pour un réseau informatique, cette adresse ip est un
identifiant unique attribué à chaque interface avec le réseau ip et associé à
une machine. Cette adresse est unicast utiliser comme adresse soucre ou de
destination. Il faudrait aussi préciser que le protocole IPV4 permet aussi
l’utilisation des adresses multicast et broadcast en dehors des adresses
unicast.

\vspace{1cm}
Exemple: adresse à valeur décimale: 212.217.0.1 => correspond sous sa forme
binaire à: 11010100.11011001.00000000.00000001
\vspace{1cm}

\subsubsection{Notion de netid et hostid}
Une adresse ipv4 est la composition de deux partie distinctes: une première
partie de l’adresse qui identifie le réseau communément appeler netid , elle
est la partie gauche de l’adresse et designe le réseau auquel appartient les
ordinateurs (hôte).  Une seconde partie de l’adresse qui identifie le numéro de
l’hôte appeler host-ID, elle est la partie droite de l’adresse et désigne
l’ensemble des hôtes du réseau.

\vspace{1cm} 
%TODO image goes here
<-----------------------4 octects------------------------------------>
        Net ID                              Host ID
   Identifiant réseau               Identifiant de l’hôte
\vspace{1cm} 


Pour connaître la limite entre netID et hostID, il faudrait connaître d’abord
le masque du réseau que nous introduirons la suite.

\textbf{ Remarque:} on distingue alors deux situations qui peuvent être soit
des échanges directes ou indirectes.  Les différentes matériels communiquent
entre eux dans la mesure qu’ils soient tous sur le même réseau IP(netid). Et
peuvent être reliée physqiuement.  Et en deuxième cas l’échange indirect ou les
matériels ne sont pas sur le même réseau, passage obligatoire à travers un
routeur pour réaliser une communication externe.   

\subsubsection{Masque de réseau}

Une adresse masque est sous forme de 32 bits, elle est utilisé pour diviser
une adresse IP en sous-réseaux et spécifier les hôtes disponibles du réseau.
Dans un masque, deux bits sont toujours affectés automatiquement. Par exemple,
dans 255.255.225.0, "0" est l'adresse de réseau assignée. Dans 255.255.255.255,
"255" est l'adresse de diffusion attribué. Le 0 et 255 sont toujours assignés
et ne peuvent pas être utilisés.


\subsubsection{Format du masque}
Il est composée de 32 bits alors même taille qu’un adresse IPV4, dans le masque
ont place les bits à 1 de manière contigus et les bits à 0 à droite.

\vspace{1cm}
{\it Exemples:\\
11111111.00000000.00000000.00000000 =   255.0.0.0\\
11111111.11111111.11111111.00000000 =   255.255.255.0\\
11110000.00000000.00000000.00000000 =   240.0.0.0 
}

\vspace{1cm}
{\it Exemple invalide:\\
11111111.01111111.00000000.00000000
}

\vspace{1cm}

\subsubsection{Calcul de l’adresse réseau et numéro de l’hôte}
Pour le calcul d’un adresse réseau on effectue un ET logique bit à bit
entre le masque de réseau et l’adresse IP. Alors on détermine:

Pour la partie réseau (netid): on effectue l’opération suivante:
    net-id <== adresse IP ET (bit à bit) Masque
Exemple: 192.168.52.0 <== 192.168.52.85 \& 255.255.255.0

Pour la partie hôte (hostid): on effectue l’opération suivante:
    host-id <== adresse IP ET (bit à bit) ~Masque
Exemple: 0.0.0.85 <== 192.168.52.85 \& 0.0.0.255

\textbf{Remarque:}
De ce fait on distingue deux adresses particulières parmi tout ceux possible,
qui ne doivent jamais être attribué à des machines:

     les bits Host-ID sont à 0 : adresse attribue qu’à un réseau.
Exemple: 192.168.10.0 / 255.255.255.0 = 192.168.10.00000000

    les bits Host-ID sont à 1 : c’est un adresse de  diffusion (broadcast),
Exemple: 172.27.255.255 / 255.255.0.0 = 172.27.1111111.11111111

Donc nous pouvons en déduire que parmi tout les adresses assignable, ces
derniers sont des adresses interdites.

\subsubsection{Notation CIDR}
Dans un premier temps nous avons vu que pour connaître l’adresse d’un
réseau il faut forcement passé par le masque, une seconde forme existe et est
connue sous le nom de notation CIDR (classless inter-domain routing) RFC 1519
ou l’adressage sans classe, ce qui veut dire qu’ici on ne tient plus compte de
l’adressage par classe; Donc aucun masque n’est fixé par rapport à une classe.
Elle s’écrit avec le numéro du réseau suivi d’un slash et le nombre de bits à 1
(en partant de la gauche) en binaire du masque sous-réseau. De nos jours cette
notation est la plus utiliser car les différentes classes utiliser sont
devenues  obsolètes.
\vspace{1cm}
Exemple: 186.52.0.0/16
\vspace{1cm}
\textbf{Remarque:}\\
cette notation CIDR ne permet pas la construction des masques réseau à trous,
alors que c’était possible dans la construction de base de IPV4 mais rarement
utilisés car fastidieux la gestion.  IPV6 intègre dés sa conception l’écriture
et l’agrégation maximale des routes introduites par CIDR. 


\subsubsection{Adresses non utilisées}
il existe des adresses non utilisable comme adresse IP pour une machine:\\
les adresses réseaux: qui correspond aux adresses qui ont tous les bits de
leur partie hostid à zéro(0);\\
les adresses de diffusion (broadcast): qui correspond aux adresses qui ont
tous les bits de leur partie hostid à un(1)\\
0.0.0.0: utilise par différentes services (table de routage, DHCP) et possède
souvent une signification particulières. \\
127.X.X.X: désigne l’ordinateur lui-même ou dite adresse de bouclage
(lookback), 127.0.0.1 pour le localhost\\
> à 223.255.255.255: pour le multicast et la recherche.\\

\subsubsection{Type d’adresse IP (Schéma dans pics sur comment c’est utilisé dans un ensemble de réseau )}
On distingue deux (2) types d’adresse IP qui sont les adresses IP publiques \& privées:
Les adresses IP privées Les adresses IP privées sont représentés par toutes les
adresses IP de classe A, B et C qui sont utilisable dans un réseau local (par
exemple le LAN) alors ce qui correspond au réseau de votre entreprise ou celle
de votre réseau domestique. D’autre part, les adresses IP privées ne sont pas
utilisable sur internet (car elles ne peuvent pas être routées sur internet),
les machines qui les utilisent ne peuvent être atteint qu’à partir de votre
réseau local. Les classes A, B et C ont chacune une correspondance de plage
d’adresses IP privées à l’intérieur de la plage globale qui a été définie par
la RFC 1918. Mais l’utilisation  de celui-ci pour inter-connecter des réseau
géante (entreprise) avec des espaces adressage qui se chevauche peut causer des
problèmes. Une adresse IP privées est librement paramétrée par l’administrateur
du réseau local.
\vspace{1cm}
Les adresses privées de la classe A: 10.0.0.0 à 10.255.255.255\\
Les adresses privées de la classe B: 172.16.0.0 à 172.31.255.255\\
Les adresses privées de la classe C: 192.168.1.0 à 192.168.255.255\\
\vspace{1cm}

Alors on vient de voir que les adresses IP privées sont utilisable uniquement
sur des réseaux locaux, tandis qu’il y a des adresses IP qui ne sont utilisées
uniquement que sur internet donc nous pouvons en déduir que c’est les adresses
IP publiques non utilisable dans un réseau local. Les routeurs (par exemple:
votre box) ont une adresse IP publique du côté d’internet, ce qui permet de
rendre votre box visible sur internet (elle répondra certainement au ping). De
plus, au moment de vos connexion sur un site web vous utilisez l’adresse
publique du serveur web. De ce fait une adresse IP publique est unique dans le
monde, ce qui n’est pas le cas dans le systèmes d’adressage des adresses IP
privées qui doivent être unique seulement dans un même réseau local mais pas au
niveau planétaire étant donné que ces adresses ne peuvent pas être routées sur
internet. Une adresse IP publique est soit acheté ou fournie par la FAI.  Les
IP publiques représentent toutes les adresses IP des classes A, B et C qui ne
font pas partie de la plage d’adresses privées de ces classes ou des exceptions
de la classe A (voir Adresse non utilisé ci-dessus).


\subsubsection{Les classes d’adresses}
Au début de la création de IPV4, maintes groupes d’adresses ont été définis
pour faciliter le routage (ou cheminement) des paquets. Structurée en 5 classes
(A,B,C,D,E) selon la valeur du première octet.

%TODO images classes %
%TODO tableau recapitulatif %

De ce fait on remarque une distribution de l’espace d’adressage selon laquelle
la classe A possède 50\% l’espace et soit 25\% pour la classe B, 12,5\% classe
C et 6,25\% pour D \& E. on peut en-déduire une mauvaise répartition de cette
espace d’adressage. 

