\section{Introduction}
\label{sec:intro}

Dans les réseaux locaux, les machines peuvent communiquer directement les unes
avec les autres par le biais d'un lien physique.\footnote{Ce lien peut directe
de machine à machine, ou indirecte: passant par d'autres équipements (switch,
hub,...).} En revanche, établir un lien entre des machines sur des réseaux
différents n'est pas aussi facile, cela pose en effet deux problèmes majeurs:
\begin{itemize}
\item Deux réseaux différents n'utilisent pas forcément la même technologie
pour transmettre des données au niveau protocolaire ou du lien physique.
\item Comme les machines ne sont pas physiquement sur le même réseau, il faut
un système d'adressage afin qu'une machine puisse joindre une autre machine
située dans un réseau différent, peu importe sa localisation.
\end{itemize}
Dans le cadre d'une communication, il est souvent pratique de diviser les
fonctionnalités nécessaire à l'échange d'information. C'est pour cette raison qu'il est
utile de définir un modèle théorique pour séparer les différentes tâches.
Aujourd'hui le standard en terme de modèle de communication est le modèle OSI
({\it Open Systems Interconnection}), qui divise en 7 couches les
fonctionnalités en question.

\bigskip
Le problème relatif à l'interconnection des réseaux (vu plus haut) est traité
par la couche 3 (nommé couche réseau) du modèle OSI.
Cette couche peux être fonctionnellement mise en oeuvre par le protocole IPv4.
C'est actuellement le protocole réseau (relative à la couche 3 du modèle OSI)
le plus utilisé et qui à permit le déploiement massive d'Internet dans le
monde.\\
Dans la suite de ce rapport, nous allons étudier le fonctionnement d'IPv4, les
possibilités qu'il offre et l'écosystème de protocoles qui gravitent autour
de lui et qui sont nécessaires à son bon fonctionnement.  Nous allons commencer
par explorer le contexte de création de ce protocole et comprendre les
motivations qui ont pousser à le concevoir.


