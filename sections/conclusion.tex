
\section{Conclusion}

Pour conclure, l'iPv4 est un protocole réseau (couche 3 du modèle OSI) qui
 a été développé dans les années 70. Les objectifs lors de son développement 
étaient d'obtenir un protocole réseau qui permet une communication efficace sur, et entre 
plusieurs réseaux informatiques. C'est un protocole basé sur la commutation de paquets, ce 
qui permet d'établir de nombreuses connexions simultanément alors qu'avant, notamment dans les réseaux téléphoniques, une connexion utilisait le lien à elle seule. 
\\
Même si des protocoles réseau existaient déjà auparavant, IPv4 a apporté une standardisation ainsi
que la commutation par paquet qui ont permit la diffusion d'internet à grande échelle.
\\
L'un des objets centraux de l'IPv4 est l'adresse IP. Elle est composée de 4 nombres allant de 0 à 255 
grâce auxquels on obtient une adresse IP écrite de cette façon : 255.0.24.35 .
C'est avec cette adresse que les interfaces/routeurs s'identifient sur internet de façon unique 
lorsqu'elle sont publiques ou de façon unique sur le réseau dans lequel elle se trouve lorsqu'elles sont locales. 
C'est donc également elle qu'on manipule pour trouver/identifier des interfaces sur le réseau/internet.
 Sa forme est restée la même au cours du temps mais la façon dont elle est allouée a été modifiée.
Il existe plusieurs types d'adresses qui sont les adresses unicast, multicast, anycast et broadcast.
 Chacune d'entre elle permet d'atteindre une ou plusieurs interfaces en fonction de certains
critères. Certaines adresses/ plages d'adresses sont également réservées pour des usages spéciaux. 
Un avantage important d'IPv4 est également la configuration automatique qui permet la mise 
en place d'un réseau sans connaissances techniques.
\\
Le second objet central d'IPv4 est le paquet IP. Un paquet est l'unité d'information réseau, donc 
de niveau 3, dans lequel sont mises les informations à transmettre. Un paquet est divisé en 2 
parties : l'en-tête et une partie donnée dans laquelle sont mises les informations à envoyer.
 L'en-tête d'un paquet IP est mis au début de ce dernier. Il donne des informations aux 
routeurs et interfaces par lesquels il transite afin d'arriver à la bonne destination. Les
 informations qui y sont contenues sont nombreuses: le protocole réseau utilisé (4 dans le
 cas d'IPv4), la taille de l'en-tête, la taille totale du paquet, un identificateur de 
fragments de même paquet, un fragment offset pour retrouver la place d'un fragment dans 
un paquet reconstitué, la durée de vie qui indique en pratique le nombre de sauts que 
peut faire un paquet IPv4, l'adresse source, l'adresse de destination et d'autres encore.
\\
Enfin, de nombreux protocoles ont été définis afin de garantir le bon fonctionnement des
 adresses IPv4. Certains protocoles servent à contrôler l'intégrité des adresses IPv4. L'ACD permet par exemple de garantir l'intégrité d'une adresse en détectant l'utilisation
de la même adresse IP par deux ou plusieurs hôtes en même temps. 
L'ICMP quant à elle permet de faire des contrôles d'erreurs et des vérifications.
D'autres permettent d'utiliser des fonctionnalités comme le protocole IGMP qui permet aux 
interfaces de s'abonner a des groupes multicast ou encore le protocole DHCP qui permet 
l'auto-configuration des interfaces. D'autres protocoles permettent finalement de simplifier
 ou de rendre plus efficace  l'utilisation d'internet comme les serveurs DNS qui font la
 traduction entre nom de domaine, comme Google.fr, et son adresse IP ou encore le PMTU 
discovery qui permet d'éviter la fragmentation de paquets.
\\
Cependant, même si au début IPv4 a été conforme aux attentes, internet a grandit de façon
 exponentielle et des problèmes auxquels ne s'attendaient pas ses concepteurs sont apparus.
Le problème le plus important d'entre eux a été l'épuisement d'adresse. En effet, internet n'était 
initialement pas destiné a un usage si répandu mais uniquement à un usage militaire et scientifique. 
Ainsi, à force de se répandre les plages d'adresses disponible s'amenuisèrent rapidement
 et c'est en février 2011 que la réserve de bloc d'adresses publiques libres IPv4 de
 l'IANA (Internet Assigned 
Numbers Authority) est arrivée à épuisement. Des techniques de contournement ont donc été
 inventées.  La plus simple a été la récupération de plages d'adresses. Une autre a été une
 modification dans l'adressage d'IPv4 en passant de la méthode classfull network à la
 méthode classless network. Ceci a permit d'éviter de gaspiller de nombreuses adresses
 IP. Enfin, à travers la traduction d'adresse (NAT : Network Adress Translation), de
 nombreuses interfaces ont pu être rassemblées sous une même adresse IP. Ainsi, on a pu
 retarder l'épuisement des adresses IP mais on n'a pas réglé le problème.
\\
Ensuite, il y a également eu d'autres plus petit problèmes concernant la sécurité et la performance qui n'étaient pas directement intégré dans IPV4. 
\\
C'est pour cela que l'IETF a décidé dans les années 90 de travailler sur un nouveau protocole 
réseau qui répondra mieux aux besoin d'internet aujourd'hui. Ce nouveau protocole qui se nomme
IPv6 ainsi que son environnement ont été développé dans les années 90 et a été publié en 1998
 dans la RFC2460. Elle apporte un espace d'adressage sur 128 bits, ce qui régla tous les soucis
 concernant l'épuisement de ce dernier. 
En plus de cela il amène une amélioration au niveau de la sécurité avec l'intégration de protocole
 tel que IPsec qui en devient partie intégrante alors qu'il n'était qu'optionnel dans IPv4. Enfin,
 IPv6 amène des améliorations au niveau des performances avec l'élimination du NAT qui enlève une
 couche supplémentaire, une simplification de l'en-tête du paquet ainsi qu'une diminution de
 traitement des routeur grâce à la fragmentation au niveau de l'émetteur et le déplacement des
 checksums dans les protocoles de niveau 4.

C'est à cause de tous ces avantages que l'IPv6 a été développé afin de remplacer l'IPv4. 
Cependant, la migration qui se déroule actuellement est lente et difficile. En effet, il 
faut faire cohabiter ces deux technologies, ce qui peut mener à des soucis de sécurité. Enfin, au 
début de 2016, le déploiement IPv6 est encore très limité car la proportion d'utilisateurs Internet en
 IPv6 est estimée à 10%. Il faudra donc encore longtemps avant que tous les utilisateurs d'internet 
sont passés à IPv6.



