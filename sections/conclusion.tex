\section{Conclusion}

Pour conclure, IPv4 est un protocole réseau (couche 3 du modèle OSI) qui a
été développé dans les années 70. Les objectifs lors de son développement
étaient d'obtenir un protocole réseau qui permet une communication efficace
entre plusieurs réseaux informatiques. C'est un protocole basé sur la
commutation de paquets, ce qui permet d'établir de nombreuses connexions
simultanément. 

\smallbreak
Même si des protocoles réseau existaient déjà auparavant, IPv4 a eu le mérite
de réussir à s'imposer comme standard dans la commutation par paquet, ce qui a
permit la diffusion d'Internet à grande échelle.

\smallbreak
L'un des objets centraux de l'IPv4 est l'adresse IP.  La forme des adresses
IPv4 est restée la même au cours du temps mais la façon dont elle est allouée a
été modifiée selon les exigences.  La taille d'un adresse IPv4 a permit
l'utilisation d'une représentation simple d'un enchainement de 4 nombres
decimaux allant de 0 à 255.  Cette représentation simple à, comme les
mécanismes de configuration automatique qui permet la mise en place d'un réseau
sans connaissances techniques: tels que APIPA, contribué à la diffusion
d'IPv4 chez le grand public.
\smallbreak

Comme on a vu dans la section \ref{sec:suiteprot}, une suite de protocoles a
été définie autour de IPv4 afin d'en garantir le bon fonctionnement et d'établir
des lien avec d'autres protocoles.

La suite de protocole IPv4 a permit de servir de base pour les futurs
implémentations standard de IPv6. A ce titre plusieurs de ces protocoles ont
fourni les principes de réalisation, pour des tâches comme les contrôles d'erreurs,
la vérification de l'intégrité des adresses, la gestion des groupes multicast,
etc..., pour IPv6.
\smallbreak


Cependant, même si au début IPv4 était conforme aux attentes, Internet a
grandit de façon exponentielle et des problèmes auxquels ne s'attendaient pas
ses concepteurs sont apparus.
Le problème le plus important d'entre eux à été l'épuisement des adresses. En
effet, Internet n'était initialement pas destiné à un usage si répandu mais
uniquement à un usage militaire et scientifique.  Ainsi, à force de se répandre
les plages d'adresses disponible s'amenuisèrent rapidement et c'est en février
2011 que la réserve d'adresses libres IPv4 est
arrivée à épuisement. Des techniques de contournement ont donc été inventées.
Ainsi, on a pu retarder l'épuisement des adresses IP mais on n'a pas réglé le
problème pour autant.

\smallbreak
C'est pour cela que l'IETF a décidé dans les années 90 de travailler sur un
nouveau protocole réseau qui répondrait mieux aux besoins de l'Internet
d'aujourd'hui. Ce nouveau protocole, qui se nomme IPv6, ainsi que son
environnement ont été développés dans les années 90 et publié en 1998 dans le
RFC2460\cite{url-RFC-IPV6}. 

\smallbreak
IPv6 simplifie l'en-tête du paquet et apporte un espace d'adressage sur 128
bits, ce qui régle tous les soucis concernant l'épuisement d'adresses.  Il
amène aussi une amélioration au niveau de la sécurité avec l'intégration de protocole
tel que IPsec et permet l'élimination d'un NAT intermédiaire. Il permet aussi
une diminution des traitements effectués par les routeur grâce à la
fragmentation des paquet qui est faite par l'émetteur, et l'élimination des checksums.


Cependant le déploiment d'IPv6 est actuellement lent et
difficile. Aujourd'hui l'utilisation d'IPv6 est encore très faible par rapport à
celle d'IPv4: début 2016, la proportion d'utilisateurs Internet utilisant IPv6
était éstimée à 10\%. 

\smallbreak

Il faudra donc encore du temps avant qu'IPv4 devienne obsolète et que les usagés d'Internet
arrêtent de l'utiliser.
