\section{Conclusion}

Pour conclure, l'iPv4 est un protocole réseau (couche 3 du modèle OSI) qui a
été développé dans les années 70. Les objectifs lors de son développement
étaient d'obtenir un protocole réseau qui permet une communication efficace
sur, et entre plusieurs réseaux informatiques. C'est un protocole basé sur la
commutation de paquets, ce qui permet d'établir de nombreuses connexions
simultanément alors qu'avant, notamment dans les réseaux téléphoniques, une
connexion utilisait le lien à elle seule. 

\smallbreak
Même si des protocoles réseau existaient déjà auparavant, IPv4 a eu le mérite
de reussir a s'imposer comme standard dans la commutation par paquet, ce qui a
permit la diffusion d'internet à grande échelle.

\smallbreak
L'un des objets centraux de l'IPv4 est l'adresse IP.  La forme des adresses
IPv4 est restée la même au cours du temps mais la façon dont elle est allouée a
été modifiée selon les exigences.  La taille d'un adresse IPv4 a permi
l'utilise d'une representation simple comme l'enchainement de 4 nombres
decimaux allant de 0 à 255.  Cela ensemble a des autres facilites, comme les
mechanisme de configuration automatique qui permet la mise en place d'un réseau
sans connaissances techniques tels que APIPA, ont contribue a la diffusion de
IPv4 dans le grand public.
\smallbreak

Comme on a vu dans la section \ref{sec:suiteproc}, une suite de protocoles a
ete definie autour de IPv4 afin d'en garantir le bon fonctionnement et etablir
le lien avec autres protocoles.

La suite de protocole IPv4 a permis de servir de base pour les futur
implémentation standard de IPv6 A ce titre plusieurs de ces protocoles ont
fourni les principes suivant lequels des taches comme les contrôles d'erreurs,
la verification de l'intégrité des adresses, la gestion des groupes multicast,
etc.. sont realisee.
\smallbreak


Cependant, même si au début IPv4 a été conforme aux attentes, internet a
grandit de façon exponentielle et des problèmes auxquels ne s'attendaient pas
ses concepteurs sont apparus.
Le problème le plus important d'entre eux a été l'épuisement d'adresse. En
effet, internet n'était initialement pas destiné a un usage si répandu mais
uniquement à un usage militaire et scientifique.  Ainsi, à force de se répandre
les plages d'adresses disponible s'amenuisèrent rapidement et c'est en février
2011 que la réserve de bloc d'adresses publiques libres IPv4 de l'IANA est
arrivée à épuisement. Des techniques de contournement ont donc été inventées.
Ainsi, on a pu retarder l'épuisement des adresses IP mais on n'a pas réglé le
problème.

\smallbreak
C'est pour cela que l'IETF a décidé dans les années 90 de travailler sur un
nouveau protocole réseau qui répondra mieux aux besoins de l'Internet
d'aujourd'hui. Ce nouveau protocole qui se nomme IPv6 ainsi que son
environnement ont été développé dans les années 90 et publié en 1998 dans le
RFC2460. 

\smallbreak
IPv6 simplifie l'en-tête du paquet et apporte un espace d'adressage sur 128
bits, ce qui régla tous les soucis concernant l'épuisement d'adresses.  Il
amène une amélioration au niveau de la sécurité avec l'intégration de protocole
tel que IPsec et perment l'élimination de la couche supplementaire du NAT ainsi
qu'une diminution des traitements effectués par les routeur grâce à la
fragmentation au niveau de l'émetteur et l'elimination des checksums.


Cependant le deploiment de IPv6 qui se déroule actuellement est lente et
difficile. Aujourd'hui l'utilise de IPv6 est encore tres faible par rapport a
ceux de IPv4: au début de 2016, la proportion d'utilisateurs Internet en IPv6
etait estimée à 10\%. 

\smallbreak

Il faudra donc encore longtemps avant que tous les utilisateurs d'internet
oublient IPv4.



